\documentclass[12pt,a4paper]{article}
\usepackage[utf8]{inputenc}
\usepackage[T1]{fontenc}
\usepackage[french]{babel}
\usepackage{geometry}
\geometry{left=2.5cm,right=2.5cm,top=2.5cm,bottom=2.5cm}
\usepackage{lmodern}
\usepackage{amsmath,amssymb}
\usepackage{graphicx}
\usepackage{xcolor}
\usepackage{hyperref}
\usepackage{caption}
\usepackage{listings}
\lstset{
	language=R,
	basicstyle=\ttfamily\small,
	keywordstyle=\color{blue}\bfseries,
	commentstyle=\color{gray},
	stringstyle=\color{red},
	frame=single,
	breaklines=true,
	showstringspaces=false
}

\title{\textbf{Jeu de données \textit{ToothGrowth} -- Description et guide d'analyse}}
\author{Travail de groupe -- AIMS 2025/26}
\date{}

\begin{document}
	\maketitle
	
	\section*{Titre}
	\textbf{L’effet de la vitamine C sur la croissance des dents chez les cobayes}
	
	\section*{A. Description (traduction)}
	La réponse est la longueur des odontoblastes (cellules responsables de la croissance des dents) mesurée chez \textbf{60 cobayes}.  
	Chaque animal a reçu une des trois doses de vitamine C (0{,}5, 1 et 2 mg/jour) administrées par l’une des deux méthodes suivantes : jus d’orange (\texttt{OJ}) ou acide ascorbique (vitamine C pure, codé \texttt{VC}).
	
	\subsection*{Format}
	Un \textit{data frame} de 60 observations sur 3 variables :
	\begin{itemize}
		\item \texttt{len} (numérique) : longueur de la dent (mm) ;
		\item \texttt{supp} (facteur) : type de supplément (\texttt{VC} ou \texttt{OJ}) ;
		\item \texttt{dose} (numérique) : dose en mg/jour (0{,}5, 1, 2).
	\end{itemize}
	
	\section*{B. Explication et objectifs possibles}
	Le jeu de données sert à étudier l'effet de la dose et du type de supplément sur la croissance dentaire. Questions typiques :
	\begin{itemize}
		\item La longueur moyenne des dents diffère-t-elle entre les doses ? (effet dose)
		\item Le type de supplément (\texttt{VC} vs \texttt{OJ}) a-t-il un effet sur la longueur ? (effet supp)
		\item Existe-t-il une interaction dose × type de supplément ?
	\end{itemize}
	
	\section*{C. Analyse recommandée (structure pour la présentation)}
	\subsection*{1. Analyse exploratoire (EDA)}
	\begin{itemize}
		\item Résumés numériques : \texttt{summary(len)} ; moyennes, médianes, écart-type par groupe (\texttt{aggregate} ou \texttt{tapply}).
		\item Graphiques : histogrammes de \texttt{len}, boxplots de \texttt{len} selon \texttt{dose} et selon \texttt{supp}, scatter/stripchart si nécessaire.
		\item Tableaux : tableau croisé des effectifs par dose et supp.
	\end{itemize}
	
	\subsection*{2. Inférence}
	\begin{itemize}
		\item Comparaisons de moyennes : tests t (paires) ou ANOVA à un facteur (dose) ; ANOVA à deux facteurs (dose et supp) pour tester interaction.
		\item Intervalles de confiance pour moyennes de chaque groupe.
		\item Régression linéaire (modèle avec \texttt{len} comme réponse, \texttt{dose} et \texttt{supp} comme facteurs) pour estimer effets et interactions.
	\end{itemize}
	
	\section*{D. Exemple de code R (à utiliser pour l'analyse)}
	\begin{lstlisting}
	%	# Charger les données (dans R les données ToothGrowth sont fournies)
		data("ToothGrowth")
		str(ToothGrowth)
		summary(ToothGrowth)
		
	%	# Résumés par groupe
		aggregate(len ~ supp + dose, data = ToothGrowth, FUN = function(x) c(mean=mean(x), sd=sd(x), n=length(x)))
		
		# Graphiques
		boxplot(len ~ dose, data = ToothGrowth, main="Len selon la dose", xlab="Dose (mg/jour)", ylab="Longueur (mm)")
		boxplot(len ~ supp, data = ToothGrowth, main="Len selon le supplément", xlab="Supplément", ylab="Longueur (mm)")
	%	# Boxplot combiné dose x supp
		interaction.plot(ToothGrowth$dose, ToothGrowth$supp, ToothGrowth$len, type="b", main="Interaction dose x supp", xlab="Dose", ylab="Longueur moyenne")
		
	%	# Test d'ANOVA (modèle à deux facteurs avec interaction)
		ToothGrowth$dose <- factor(ToothGrowth$dose) # traiter dose comme facteur si souhaité
		mod <- aov(len ~ supp * dose, data = ToothGrowth)
		summary(mod)
		
		# Si dose traitée comme variable numérique (effet linéaire possible)
		mod2 <- lm(len ~ as.numeric(as.character(dose)) * supp, data = ToothGrowth)
		summary(mod2)
		
        #Tests t : comparaison entre supp à dose = 0.5 par exemple
		subset05 <- subset(ToothGrowth, dose == 0.5)
		t.test(len ~ supp, data = subset05)
		
		# Intervalles de confiance pour moyennes (ex. pour OJ)
		library(dplyr)
		ToothGrowth %>% filter(supp == "OJ") %>% summarise(mean_len = mean(len), sd = sd(len), n=n(), 
		se = sd/sqrt(n), 
		ci_low = mean_len - qt(0.975, n-1)*se, 
		ci_high = mean_len + qt(0.975, n-1)*se)
	\end{lstlisting}
	
	\section*{E. Exemples d'interprétation (à présenter)}
	\begin{itemize}
		\item Si l'ANOVA montre un effet significatif de la dose : « la longueur moyenne augmente significativement avec la dose. »
		\item Si l'effet de \texttt{supp} est significatif : « le type de supplément (OJ vs VC) influence la croissance. »
		\item Si l'interaction est significative : « l'effet de la dose dépend du type de supplément (par ex. OJ produit une plus grande augmentation à certaines doses). »
	\end{itemize}
	
	\section*{F. Conseils pratiques pour la présentation (12 minutes)}
	\begin{enumerate}
		\item 0--2 min : Introduction et description du jeu de données.
		\item 2--6 min : EDA — graphiques clés et résumés numériques (1--2 figures).
		\item 6--11 min : Inférence — poser la question principale, méthode choisie, résultats (tableau ou sortie ANOVA / test t), interprétation.
		\item 11--12 min : Conclusion concise et recommandations (limites de l'étude, perspectives).
	\end{enumerate}
	
	\vspace{0.5cm}
	\noindent\textbf{Bonne analyse !} Si tu veux, je peux : 
	\begin{itemize}
		\item te fournir le code R complet exécuté avec sorties et graphiques (si tu veux un PDF des graphiques), 
		\item ou convertir ce document en PDF/Beamer pour la présentation.
	\end{itemize}
	
	
	\section*{Instructions}
	\begin{itemize}
		\item À chaque groupe a été attribué un jeu de données différent. Tous les jeux de données comprennent :
		\begin{itemize}
			\item au moins une \textbf{variable catégorielle} qui divise les données en deux groupes ou plus ;
			\item quelques \textbf{variables continues}.
		\end{itemize}
		\item Les détails sur votre jeu de données spécifique sont fournis dans la feuille \texttt{DataDescription} du tableur qui vous a été attribué.
		\item Votre tâche pour la présentation est d'\textbf{analyser les données}. Des orientations sur les questions à traiter sont fournies ci-dessous.
		\item La présentation doit être divisée en deux parties :
		\begin{enumerate}
			\item \textbf{Analyse exploratoire des données (EDA)} : statistiques descriptives du jeu de données.
			\item \textbf{Inférence} : questions statistiques de votre choix.
		\end{enumerate}
		Veillez à relier correctement ces deux parties dans votre présentation.
	\end{itemize}
	
	\section*{Directives générales}
	Voici quelques lignes directrices sur le type de questions que vous pouvez explorer. Adaptez ou étendez selon votre jeu de données.
	
	\subsection*{1. Analyse exploratoire des données (EDA)}
	\begin{itemize}
		\item Utilisez à la fois des \textbf{résumés numériques} et des \textbf{résumés graphiques} pour décrire vos données.
		\item Lorsque vous présentez des résumés numériques, réfléchissez si un \textbf{tableau} ne les rendrait pas plus clairs.
		\item Si vous créez plusieurs graphiques pour la même variable, mettez-les en relation entre eux et avec les résultats numériques.
		\item Personnalisez vos graphiques (étiquettes d'axes, taille des polices, couleurs, légendes, etc.). Les graphiques doivent être clairs et informatifs pour l'auditoire.
	\end{itemize}
	
	\subsection*{2. Inférence}
	\begin{itemize}
		\item Dans votre présentation, \textbf{énoncez clairement} les questions que vous voulez répondre et justifiez les méthodes choisies.
		\item Vous pouvez construire des \textbf{intervalles de confiance} pour certains paramètres et comparer les résultats entre groupes si pertinent.
		\item Les intervalles de confiance s'appliquent aussi aux \textbf{proportions} — vérifiez la pertinence pour votre jeu de données.
		\item Effectuez des \textbf{tests d'hypothèses} pour comparer des groupes lorsque c'est approprié.
		\item Si pertinent, explorez les \textbf{relations entre variables numériques} (corrélation, régression linéaire, ...).
		\item Indiquez les formules utilisées et présentez les résultats ; préparez-vous à les commenter et à expliquer leur sens.
	\end{itemize}
	
	\paragraph{Remarque importante :} Les points ci-dessus sont des suggestions. Les questions les plus pertinentes dépendront de votre jeu de données. N'en faites pas trop : vous disposez de \textbf{12 minutes} pour la présentation — privilégiez la clarté.
	
	\section*{Considérations supplémentaires}
	\begin{itemize}
		\item Pendant l'EDA, vérifiez si l'une des distributions étudiées convient visuellement à vos données (superposer une densité théorique sur un histogramme).
		\item Si une distribution semble convenir, complétez le contrôle visuel par un contrôle numérique : comparez quelques percentiles observés avec les percentiles théoriques correspondants.
	\end{itemize}
	
	\section*{Note finale}
	Privilégiez la clarté et l'interprétation plutôt que la quantité. Une analyse concise et bien raisonnée est plus efficace qu'une présentation surchargée.
	
\end{document}
